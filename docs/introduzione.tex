\section{Introduzione}
In questa relazione si analizzano alcune possibilità di implementazione parallela dell'algoritmo \textbf{Gift Wrapping}\footnote{\url{https://en.wikipedia.org/wiki/Gift\_wrapping\_algorithm}}.
È stato scelto di realizzarne una versione usando \textbf{OpenMP} e una usando \textbf{MPI}.

Per poter facilmente confrontare il risultato delle diverse soluzioni, ogni implementazione trova il \emph{convex hull} con il minor numero di punti;
la versione seriale è stata quindi leggermente modificata per rimuovere dall'\emph{hull} punti collineari consecutivi\footnote{Questa piccola modifica ha comportanto un inaspettato aumento della velocità di esecuzione nella versione parallela, dovuto probabilmente a qualche ottimizazione del compilatore o al verificarsi di qualche fortunata condizione}.
 
Verranno utilizzati dei grafici per mostrare le performance delle diverse soluzioni:
in ognuno saranno riportati i dati (\emph{speedup} o \emph{tempi di esecuzione} a seconda dei casi) per \textbf{tutti} gli input forniti,
in modo da poter verificare il comportamento sia sulle diverse tipologie di input (cerchi, distribuzioni gaussiane...) sia sulle diverse dimensioni (numero di punti).

Tutti i tempi vengono misurati sul server \texttt{isi-raptor03} fornito da \emph{UNIBO}, realizzando \textbf{10} diverse misurazioni, scartando le \textbf{2} più veloci e le \textbf{2} più lente e facendo la media delle rimanenti;
questo per cercare di ottenere dati il più possibile accurati e di escludere eventuali oscillazioni nel carico del server.

Per calcolare lo \emph{speedup} è stata utilizzata la formula
\begin{equation}
    \frac{T_{parallel}(1)}{T_{parallel}(P)}
\end{equation}
in cui $T_{parallel}(P)$ è il tempo di esecuzione con $P$ processori.
