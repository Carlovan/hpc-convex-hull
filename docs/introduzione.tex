\section{Introduzione}
In questa relazione si analizzano alcune possibilità di implementazione parallela dell'algoritmo \textbf{Gift Wrapping}\footnote{\url{https://en.wikipedia.org/wiki/Gift_wrapping_algorithm}}.
È stato scelto di realizzarne una versione usando \textbf{OpenMP} e una usando \textbf{MPI}.

Per poter facilmente confrontare il risultato delle diverse soluzioni, ogni implementazione trova il \emph{convex hull} con il minor numero di punti;
la versione seriale è stata quindi leggermente modificata per rimuovere dall'\emph{hull} punti collineari consecutivi\footnote{Questa piccola modifica ha comportanto un inaspettato aumento della velocità di esecuzione nella versione parallela, dovuto probabilmente ad del compilatore o al verificarsi di qualche fortunata condizione}.
 
Per calcolare lo \emph{speedup} è stata utilizzata la formula
\begin{equation}
    \frac{1}{|I|} \sum_{i \in I}{} \frac{T_{parallel}(1)}{T_{parallel}(P)}
\end{equation}
in cui $I$ è l'insieme degli input, $T_{parallel}(P)$ è il tempo di esecuzione con $P$ processori; i tempi sono stati ottenuti con 5 diverse misurazioni,
rimuovendo la più lenta e la più veloce e facendo la media delle rimanenti.
