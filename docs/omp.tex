\section{Implementazioni OpenMP}
Per parallelizzare l'algoritmo usando \emph{OpenMP} sono state realizzate alcune versioni in modo da valutare l'efficacia di diversi approcci strumenti.
In particolare è risultato evidente che impostando la variabile d'ambiente \texttt{OMP_PROC_BIND=true} si ottengono risultati migliori sia dal punto di vista dell'efficienza , ma soprattutto da quello della \textbf{ripetibilità}: senza l'utilizzo della variabile esecuzione successive dello stesso programma terminavano spesso con tempi molto diversi.

Analizzando il funzionamento dell'algoritmo si può dire che ad ogni iterazione viene aggiunto all'\emph{hull} il punto più esterno rispetto all'ultimo inserito $P_l$;
$P_a$ è più esterno di $P_b$ se la spezzata $P_lP_aP_b$ è orientata in senso orario.

Tornando nel mondo parallelo è possibile eseguire questa operazione con il \textbf{pattern reduction}, per il quale OpenMP offre una direttiva dedicata.
Di seguito sono riportati i tempi parallelizzando la ricerca del prossimo punto $P_next$ usando la direttiva \texttt{reduce} con una operatore personalizzato,
dichiarato con \texttt{\#pragma omp declare reduction}:

